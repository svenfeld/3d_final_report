\documentclass[conference]{IEEEtran}
%\IEEEoverridecommandlockouts
% The preceding line is only needed to identify funding in the first footnote. If that is unneeded, please comment it out.
\usepackage{cite}
\usepackage{amsmath,amssymb,amsfonts}
\usepackage{algorithmic}
\usepackage{graphicx}
\usepackage{textcomp}
\usepackage{xcolor}
\usepackage[nolist]{acronym}
\usepackage[official]{eurosym}
\usepackage{hyperref}
\usepackage{enumitem}
\usepackage{float}


\def\BibTeX{{\rm B\kern-.05em{\sc i\kern-.025em b}\kern-.08em
		T\kern-.1667em\lower.7ex\hbox{E}\kern-.125emX}}

% Used for displaying a sample figure. If possible, figure files should
% be included in EPS format.
%
% If you use the hyperref package, please uncomment the following line
% to display URLs in blue roman font according to Springer's eBook style:
\renewcommand\UrlFont{\color{blue}\rmfamily}

\begin{document}
	
	
	%
	\title{}
	
	\author{}
	\author{\IEEEauthorblockN{Joni, Sabrina ,Sven}
		\IEEEauthorblockA{\textit{Department} \\
			\textit{University}\\
			Location}
	}
	
	\maketitle
	
	\begin{abstract}
		
		Virtual Reality (VR) devices have undergone significant changes over the last years, transforming the way users interact with virtual environments. This paper presents a systematic review of VR and MR devices developed in the past five years, with a specific focus on their application within the entertainment and gaming industry. Our research entails an exhaustive systematic review of literature leading to the final selection of 41 papers. We build upon the taxonomy developed by Adilkhanov et al.~\cite{Adilkhanov22}, adapting it to categorize the latest VR devices. Our findings reveal the diverse landscape of VR devices designed for gaming and entertainment. Subcategories highlight the complex choices available to users. This study reveals the absence of a standardized devices in the VR domain, emphasizing the need for further research to streamline and enhance user experiences in virtual environments.
		
	\end{abstract}
	
	\begin{IEEEkeywords}
		VR, MR, Devices, Categorisation, Systematic Review
	\end{IEEEkeywords}
	
	
	\section{Introduction}
\label{intro}

%------------ topic & relevance --------------
The realm of haptic technology has witnessed remarkable growth in the past decade, driven by applications such as haptic robot teleoperation, virtual reality (VR), and mixed reality (MR). However, there remains substantial work to be done in enabling a fully immersive interaction with objects in virtual environments (VE). Achieving realistic object manipulation, including the perception of textures, shape, weight, softness, and temperature, is essential to enhance the immersion of users in the virtual world. Advancements in devices used in VR and MR are pivotal in complementing the visual and haptic experiences typically provided by several kinds of devices. Devices play a crucial role in providing feedback, enabling users to experience a sense of touch, often extending to the creation of haptic illusions. Some devices can also receive information from the virtual environment (VE) and relay feedback to the user, simultaneously transmitting the sensed position and force data of the user back to the VE.

%---------------- the problem this paper addresses -------------------
This paper addresses a significant problem in the field of 3D interface devices, specifically within the realms of Mixed Reality (MR) and Virtual Reality (VR), with a particular focus on the gaming and entertainment industry. As the adoption of MR and VR technologies continues to grow, there is a pressing need for a comprehensive understanding of the latest developments in these domains. The problem at hand lies in the lack of a structured categorization framework for 3D interface devices, making it challenging for researchers and stakeholders to navigate and comprehend the rapidly evolving landscape of MR and VR technologies in entertainment. Without a clear taxonomy, it becomes difficult to identify the most frequently used devices and their applications within specific fields of the entertainment and gaming sector. This research seeks to address this problem by systematically categorizing and exploring the latest developed devices.
%---------------- outline of the remaining paper --------------------

Following this, in chapter~\ref{sec:rel_work} related work is being presented. Afterwards, in chapter~\ref{sec:method} the methodology section outlines the systematic approach used for the literature review. Discussing the search strategy, research questions, data collection, and the criteria for inclusion and exclusion. The paper then delves in chapter~\ref{sec:results} into the results, where the findings of the literature review are presented and the categorization of devices used in the domains of gaming and entertainment is being given. The importance of subcategorization is highlighted, shedding light on the nuances of user interactions with these devices. Finally, the paper concludes by summarizing the key takeaways from the study, emphasizing the significance of a structured categorization framework for researchers and industry professionals.

	\section{Related Work}
\label{sec:rel_work}

In recent years, the landscapes of Virtual Reality (VR) and Augmented Reality (AR) have undergone significant transformations. Central to this evolution is haptic technology, which has redefined user immersion by facilitating tangible interactions within virtual environments. While we aim to focus on the advancements over the past five years, a foundational base for our study was the work of Adilkhanov et al~\cite{Adilkhanov22}. We used their taxonomy as a foundation of our categorization.

Adilkhanov et al. presented a unique and clear categorization system for haptic devices based on wearability. This taxonomy provided a easy to understand framework, making it easier to understand the diverse range of devices and their respective applications. By grouping devices by how they're worn, they laid a clear groundwork for classifying the haptic domain. This was also easy way to us how to define the equipment in the easy way not making it too hard to understand. It is like a filter that defines what to leave out from our results.

In the work of Adilkhanov et al., they used a systematic way to define relevant papers from major academic databases, filtering for those that introduced innovative concepts or offered new insights into haptic feedback and device modifications, and that were published from 2010 - 2021.

	\section{Methodology}
\label{sec:method}
In the next sections, we describe our methodology in more detail.

\subsection{Search Strategy}

Our literature review followed a systematic approach presented by Kitchenham~\cite{Kitchenham06}. The first step in our methodology was to search for existing literature reviews that focus the same topic. We found the work of Adilkhanov~\cite{Adilkhanov22}. However, there literature review is general. We in comparison want to focus on the entertainment industry specifically. Nevertheless, we used the taxonomy created by Adilkhanov as a first stepping stone to categorize our literature review. We created a carefully crafted search term. The search term was designed to encompass the relevant literature to answer our research question.

(gaming OR entertainment OR recreation OR games) AND ( VR OR virtual reality OR MR OR mixed reality) AND (wearables OR controller OR devices)


\subsection{Research Question}
What are the developed devices in the last 5 years in Mixed Reality (MR) and Virtual Reality (VR), specifically focusing on entertainment? 

What are the most common devices researched in the field of gaming and how could they be categorized?

\subsection{Data Collection}
We conducted our literature search by querying the IEEE Xplore library\footnote{https://ieeexplore.ieee.org/Xplore/home.jsp}, a well-known online database for computer science literature. This database was chosen due to its reputation and popularity among computer science studies. The initial search yielded a total of 699 results, representing potential sources for our review.

\subsection{Inclusion and Exclusion Criteria}
To narrow down our search results, we established a set of inclusion and exclusion criteria. These criteria were designed to ensure that the selected papers were both relevant and of high quality. Our inclusion criteria consisted of factors such as the publication date, the alignment with the research topic and that a device was described in the study. On the other hand, exclusion criteria was: papers were not written in English, were duplicate publications or had less than 4 pages.


\subsection{Title Screening}
The next step involved a preliminary title screening of the 699 results. To distribute the workload efficiently, each member of our research team was assigned approximately one-third of the total results. After the title screening, 573 papers were excluded, as they did not meet our research objectives or failed to satisfy the inclusion criteria, leaving us with a reduced set of papers for further analysis.


\subsection{Full Paper Evaluation}
The remaining papers, a total of 126 articles, underwent a  more thorough examination. In this phase, we evaluated the content and relevance of each paper to our research question. These assessments helped us identify the studies that would contribute with devices to our literature review.

\subsection{Quality Assurance}
To ensure the validity of our review, we implemented a process of double-checking. The papers that were included in our final selection were reviewed by another researcher of our team to confirm their suitability. Similarly, the papers that were excluded during the full paper evaluation stage were reviewed again by a different team member to reduce the risk of inadvertently excluding valuable contributions.

\subsection{Final Selection}
After our screening and quality assurance, a final set of 41 papers was identified as suitable for our literature review. These selected papers met our predefined inclusion criteria, demonstrated relevance to our research question by presenting a device related to VR.

This methodology provided a systematic and transparent approach to conducting our literature review, facilitating the selection of research papers for our analysis.








Example Figure:

%\begin{figure}[htbp]
%	\includegraphics[width=\columnwidth]{figures/definition_hierarchy.pdf}
%%	\label{fig:def_hierarchy}
%\end{figure}


	\section{Results}
\label{sec:results}

In this section, ...


\subsection{Qualitative Analysis}
\label{qual_res}

\textbf{TODO: Overview of the final code book, put full code book into appendix}

\textbf{TODO: Explain results of the coding}


\subsection{Quantitative Analysis}
\label{quant_res}

\textbf{TODO: Explain the different quantitative results, like demographics etc.}
	\section{Conclusion}
\label{sec:conclusion}

In conclusion, this paper delved into the rapidly evolving landscape of VR and MR devices. As the entertainment and gaming industry embraces the potential of immersive technologies, it becomes paramount to categorize and understand the latest developments in VR hardware. With a specific focus on the past five years, this study explored the latest innovations within that field. The research questions were addressed systematically through an in-depth literature review, resulting in the identification and categorization of devices.

Our methodology, inspired by Adilkhanov et al.'s~\cite{Adilkhanov22} taxonomy, helped structure the diverse field of devices, emphasizing wearability as a crucial categorization factor. Notably, our findings indicate that there is no universally established category for VR devices, suggesting a continued evolution in this domain. The subcategorization of devices also underscores the importance of tailoring solutions to specific user needs, promoting enhanced usability and user satisfaction.

As the VR industry continues to expand, with new devices continuously emerging, this research serves as a foundational guide for both researchers and industry stakeholders. By offering a clear and structured overview of VR devices, we aim to support the development of innovative and immersive technologies, ultimately enhancing user experiences.

We anticipate that future studies will further contribute to this evolving field. With the foundation laid in this paper, we encourage continued exploration and innovation using the developed in-depth categorisation.
	
	\bibliographystyle{alpha}
	\bibliography{literature}
	
	%\appendix
	%\input{Content/6 - Appendix}
	
\end{document}

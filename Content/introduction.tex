\section{Introduction}
\label{intro}

%------------ topic & relevance --------------
The realm of haptic technology has witnessed remarkable growth in the past decade, driven by applications such as haptic robot teleoperation, virtual reality (VR), and mixed reality (MR). However, there remains substantial work to be done in enabling a fully immersive interaction with objects in virtual environments (VE). Achieving realistic object manipulation, including the perception of textures, shape, weight, softness, and temperature, is essential to enhance the immersion of users in the virtual world. Advancements in devices used in VR and MR are pivotal in complementing the visual and haptic experiences typically provided by several kinds of devices. Devices play a crucial role in providing feedback, enabling users to experience a sense of touch, often extending to the creation of haptic illusions. Some devices can also receive information from the VE and relay feedback to the user, simultaneously transmitting the sensed position and forcing data of the user back to the VE.

%---------------- the problem this paper addresses -------------------
This paper addresses a significant problem in the field of 3D interface devices, specifically within the realms of MR and VR, with a particular focus on the gaming and entertainment industry. As the adoption of MR and VR technologies continue to grow, there is a pressing need for a comprehensive understanding of the latest developments in these domains. The problem at hand lies in the lack of a structured categorization framework for 3D interface devices, making it challenging for researchers and stakeholders to navigate and comprehend the rapidly evolving landscape of MR and VR technologies in entertainment. Without a clear taxonomy, it becomes difficult to identify the most frequently used devices and their applications within specific fields of the entertainment and gaming sector. This research seeks to address this problem by systematically categorizing and exploring the latest developed devices.
%---------------- outline of the remaining paper --------------------

Following this, in Chapter~\ref{sec:rel_work} related work is presented. Afterwards, in Chapter~\ref{sec:method} the methodology section outlines the systematic approach used for the literature review. Discussing the search strategy, research questions, data collection, and the criteria for inclusion and exclusion. The paper then delves in Chapter~\ref{sec:results} into the results, where the findings of the literature review are presented and the categorization of devices used in the domains of gaming and entertainment is given. The importance of subcategorization is highlighted, shedding light on the nuances of user interactions with these devices. Finally, the paper concludes by summarizing the key takeaways from the study, emphasizing the significance of a structured categorization framework for researchers and industry professionals.

\section{Related Work}
\label{sec:rel_work}

In recent years, the landscapes of Virtual Reality (VR) and Augmented Reality (AR) have undergone significant transformations. Central to this evolution is haptic technology, which has redefined user immersion by facilitating tangible interactions within virtual environments. While we aim to focus on the advancements over the past five years, a foundational base for our study was the work of Adilkhanov et al~\cite{Adilkhanov22}. We used their taxonomy as a foundation of our categorization.

Adilkhanov et al. presented a unique and clear categorization system for haptic devices based on wearability. This taxonomy provided a easy to understand framework, making it easier to understand the diverse range of devices and their respective applications. By grouping devices by how they're worn, they laid a clear groundwork for classifying the haptic domain. This was also easy way to us how to define the equipment in the easy way not making it too hard to understand. It is like a filter that defines what to leave out from our results.

In the work of Adilkhanov et al., they used a systematic way to define relevant papers from major academic databases, filtering for those that introduced innovative concepts or offered new insights into haptic feedback and device modifications, and that were published from 2010 - 2021.

The first survey on haptic interfaces and devices, and on their applications was written by Laycock and Day , who also examined how haptic feedback was combined with visual display devices (e.g., virtual reality walls and workbenches), so as to improve the immersive experience. This was the foundation for the combination for visual and haptic devices.

Hayward and coauthors presented a foundational classification of haptics in human-computer interfaces, covering human kinesthetic, tactile sensing, and existing haptic devices. Subsequent reviews have explored various aspects of haptic systems, including different taxonomies, design challenges, and specific device categories like wearable haptics, glove-type wearables, and haptic gloves classified by design.

Culbertson et al. delved into the technologies that simulate artificial touch sensations. Their review emphasized the design, control, and application of noninvasive haptic devices, introducing a taxonomy that categorizes these systems into three main types: graspable, wearable, and touchable. Within each category, they explored various haptic feedback mechanisms. In a related work, Wu and Culbertson innovated a haptic forearm sleeve using pneumatic actuation, which gives the wearer an illusion of lateral motion on the arm, achieved by a series of interconnected pneumatic actuators that create a continuous point of pressure.
\section{Related Work}
\label{sec:rel_work}

In recent years, the landscapes of Virtual Reality (VR) and Augmented Reality (AR) have undergone significant transformations. Central to this evolution is haptic technology, which has redefined user immersion by facilitating tangible interactions within virtual environments. While we aim to focus on the advancements over the past five years, a groundwork for our study is the work of Adilkhanov et al~\cite{Adilkhanov22}. We used their taxonomy as a foundation of our categorization, which we discuss in Chapter in detail~\ref{sec:results}.

Adilkhanov et al. presented a unique and clear categorization system for haptic devices based on wearability. This taxonomy provided a easy to understand framework, making it easier to interpret the diverse range of devices and their respective applications. By grouping the devices by how they are worn, they laid a clear groundwork for classifying the haptic domain. In their work they used a systematic way to define relevant papers from major academic databases, filtering for those that introduced innovative concepts or offered new insights into haptic feedback and device modifications, and that were published from 2010 - 2021. 

Laycock and Day~\cite{Layvock03} summarized the devices devices being developed back in 2003. They examined into the integration of haptic feedback with visual display devices, such as virtual reality walls and workbenches, to enhance the overall immersive experience. This can be seen as one of the foundations for the combination for visual and haptic devices.

Hayward et al.~\cite{Hayward04c} presented a foundational classification of haptics in human-computer interfaces, covering human kinesthetic, tactile sensing, and existing haptic devices. Subsequent reviews have explored various aspects of haptic systems, including different taxonomies, design challenges, and specific device categories like wearable haptics, glove-type wearables, and haptic gloves classified by design.

Culbertson et al.~\cite{Culbertson18} delved into the technologies that simulate artificial touch sensations. Their review emphasized the design, control, and application of non-invasive haptic devices, introducing a taxonomy that categorizes these systems into three main types: graspable, wearable, and touchable. Within each category, they explored various haptic feedback mechanisms. In a related work, Wu and Culbertson~\cite{WuC19} innovated a haptic forearm sleeve using pneumatic actuation, which gives the wearer an illusion of lateral motion on the arm, achieved by a series of interconnected pneumatic actuators that create a continuous point of pressure.

This chapter highlighted some of academic literature researching VR and MR devices. The sources track the progressive advancements and innovations within the field. This chapter underlines the depth of research that has been conducted to evaluate and characterize these devices. Still there is now commonly settled categorization for devices used in this space.
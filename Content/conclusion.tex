\section{Conclusion}
\label{sec:conclusion}

In conclusion, this paper delved into the rapidly evolving landscape of VR and MR devices. As the entertainment and gaming industry embraces the potential of immersive technologies, it becomes paramount to categorize and understand the latest developments in VR hardware. With a specific focus on the past five years, this study explored the latest innovations within that field. The research questions were addressed systematically through an in-depth literature review, resulting in the identification and categorization of devices.

Our methodology, inspired by Adilkhanov et al.'s~\cite{Adilkhanov22} taxonomy, helped structure the diverse field of devices, emphasizing wearability as a crucial categorization factor. Notably, our findings indicate that there is no universally established category for VR devices, suggesting a continued evolution in this domain. The sub-categorization of devices also underscores the importance of tailoring solutions to specific user needs, promoting enhanced usability and user satisfaction.

As the VR industry continues to expand, with new devices continuously emerging, this research serves as a foundational guide for both researchers and industry stakeholders. By offering a clear and structured overview of VR devices, we aim to support the development of innovative and immersive technologies, ultimately enhancing user experiences.

We anticipate that future studies will further contribute to this evolving field. With the foundation laid in this paper, we encourage continued exploration and innovation using the developed in-depth categorisation.
\section{Methodology}
\label{sec:method}
In the next sections, we describe our methodology in more detail.

\subsection{Search Strategy}

Our literature review followed a systematic approach presented by Kitchenham~\cite{Kitchenham06}. The first step in our methodology was to search for existing literature reviews that focus the same topic. We found the work of Adilkhanov~\cite{Adilkhanov22}. However, there literature review is general. We in comparison want to focus on the entertainment industry specifically. Nevertheless, we used the taxonomy created by Adilkhanov as a first stepping stone to categorize our literature review. We created a carefully crafted search term. The search term was designed to encompass the relevant literature to answer our research question.

(gaming OR entertainment OR recreation OR games) AND ( VR OR virtual reality OR MR OR mixed reality) AND (wearables OR controller OR devices)


\subsection{Research Question}
What are the developed devices in the last 5 years in Mixed Reality (MR) and Virtual Reality (VR), specifically focusing on entertainment? 

What are the most common devices researched in the field of gaming and how could they be categorized?

\subsection{Data Collection}
We conducted our literature search by querying the IEEE Xplore library\footnote{https://ieeexplore.ieee.org/Xplore/home.jsp}, a well-known online database for computer science literature. This database was chosen due to its reputation and popularity among computer science studies. The initial search yielded a total of 699 results, representing potential sources for our review.

\subsection{Inclusion and Exclusion Criteria}
To narrow down our search results, we established a set of inclusion and exclusion criteria. These criteria were designed to ensure that the selected papers were both relevant and of high quality. Our inclusion criteria consisted of factors such as the publication date, the alignment with the research topic and that a device was described in the study. On the other hand, exclusion criteria was: papers were not written in English, were duplicate publications or had less than 4 pages.


\subsection{Title Screening}
The next step involved a preliminary title screening of the 699 results. To distribute the workload efficiently, each member of our research team was assigned approximately one-third of the total results. After the title screening, 573 papers were excluded, as they did not meet our research objectives or failed to satisfy the inclusion criteria, leaving us with a reduced set of papers for further analysis.


\subsection{Full Paper Evaluation}
The remaining papers, a total of 126 articles, underwent a  more thorough examination. In this phase, we evaluated the content and relevance of each paper to our research question. These assessments helped us identify the studies that would contribute with devices to our literature review.

\subsection{Quality Assurance}
To ensure the validity of our review, we implemented a process of double-checking. The papers that were included in our final selection were reviewed by another researcher of our team to confirm their suitability. Similarly, the papers that were excluded during the full paper evaluation stage were reviewed again by a different team member to reduce the risk of inadvertently excluding valuable contributions.

\subsection{Final Selection}
After our screening and quality assurance, a final set of 41 papers was identified as suitable for our literature review. These selected papers met our predefined inclusion criteria, demonstrated relevance to our research question by presenting a device related to VR.

This methodology provided a systematic and transparent approach to conducting our literature review, facilitating the selection of research papers for our analysis.








Example Figure:

%\begin{figure}[htbp]
%	\includegraphics[width=\columnwidth]{figures/definition_hierarchy.pdf}
%%	\label{fig:def_hierarchy}
%\end{figure}


\section{Methodology}
\label{sec:method}
In the next sections, we describe our methodology in more detail.

\subsection{Search Strategy}

Our literature review followed a systematic approach presented by Kitchenham~\cite{Kitchenham06}. The first step in our methodology was to search for existing literature reviews that focus the same topic. We found the work of Adilkhanov~\cite{Adilkhanov22}. However, there literature review is general. We in comparison want to focus on the entertainment industry specifically. Nevertheless, we used the taxonomy created by Adilkhanov as a first stepping stone to categorize our literature review. We created a well-defined search term. The search term was carefully crafted to encompass the relevant literature pertaining to our research question. This step ensured that the search results were focused and pertinent to the subject matter.

Search term (gaming OR entertainment OR recreation OR games) AND ( VR OR virtual reality OR MR OR mixed reality) AND (wearables OR controller OR devices)


\subsection{Research Question}


\subsection{Data Collection}
We conducted our literature search by querying the IEEE Xplore library, a renowned repository for scientific and technical literature. This database was chosen due to its reputation for containing a wealth of research in the field of interest. The initial search yielded a total of 699 results, representing potential sources for our review.

\subsection{Inclusion and Exclusion Criteria}
To narrow down our search results, we established a set of inclusion and exclusion criteria. These criteria were designed to ensure that the selected papers were both relevant and of high quality. Our inclusion criteria consisted of factors such as the publication date, the source's credibility, and the alignment with the research topic. On the other hand, exclusion criteria included papers that did not meet our inclusion criteria, were not written in English, or were duplicate publications.


\subsection{Title Screening}
The next step involved a preliminary title screening of the 699 results. To distribute the workload efficiently, each member of our research team was assigned approximately one-third of the total results. After the title screening, 573 papers were excluded, as they did not meet our research objectives or failed to satisfy the inclusion criteria, leaving us with a reduced set of papers for further analysis.


\subsection{Full Paper Evaluation}
The remaining papers, a total of [insert number here] articles, underwent a thorough and comprehensive examination. In this phase, we evaluated the content, methodology, and relevance of each paper to our research question. These assessments helped us identify the most pertinent studies that would contribute significantly to our literature review.

\subsection{Quality Assurance}
To ensure the rigor and validity of our review, we implemented a process of double-checking. The papers that were included in our final selection were reviewed independently by another researcher to confirm their suitability and quality. Similarly, the papers that were excluded during the full paper evaluation stage were subjected to a second round of review by a different team member to reduce the risk of inadvertently excluding valuable contributions.

\subsection{Final Selection}
After rigorous screening and quality assurance, a final set of [insert number here] papers was identified as suitable for inclusion in our literature review. These selected papers met our predefined inclusion criteria, demonstrated relevance to our research question, and had been verified through the double-checking process, ensuring the robustness of our literature review.

This methodology provided a systematic and transparent approach to conducting our literature review, facilitating the selection of high-quality and pertinent research papers for our analysis. It helped us to ensure the credibility and reliability of the sources included in our study.








Example Figure:

%\begin{figure}[htbp]
%	\includegraphics[width=\columnwidth]{figures/definition_hierarchy.pdf}
%%	\label{fig:def_hierarchy}
%\end{figure}

